\RequirePackage{luatex85}

% for notes environment
\usepackage{xsavebox}
\usepackage{hyperref}
\usepackage{graphicx}
\usepackage{luatexja}
\usepackage[hiragino-pro,deluxe,nfssonly,jis2004]{luatexja-preset}
\usepackage{fontspec}
\usepackage{epigraph}
\usepackage{etoolbox}
\usepackage{tikz}
\usepackage{framed}
\usepackage{mathtools}
\usepackage{listings}
\usepackage{libertine}
\usepackage[libertine]{newtxmath}
\usepackage{bxcoloremoji}
\usepackage{xcolor}
\usepackage{diagbox}
\usepackage{caption}
\usepackage{appendixnumberbeamer}
\usepackage{xpatch}
\usepackage{multicol}
\usepackage{amsmath}
\usepackage{amsthm}
\usepackage{bxtexlogo}
\bxtexlogoimport{SATySFi}

\usetikzlibrary{fit}

\setmonofont{CMU Typewriter Text}

\definecolor{links}{HTML}{2A1B81}
\hypersetup{colorlinks,linkcolor=,urlcolor=links}

\usetheme{Boadilla}
\usecolortheme{seahorse}
% \usefonttheme{serif}


\xpatchcmd{\itemize}
  {\def\makelabel}
  {\ifnum\@itemdepth=1\relax
     \setlength\itemsep{1.2ex}% separation for first level
   \else
     \ifnum\@itemdepth=2\relax
       \setlength\itemsep{0.8ex}% separation for second level
       \setlength\topsep{1.2ex}
     \else
       \ifnum\@itemdepth=3\relax
         \setlength\itemsep{0.05ex}% separation for third level
         \setlength\topsep{0.8ex}
   \fi\fi\fi\def\makelabel
  }
 {}
 {}

\setbeamercolor{page number in head/foot}{bg=blue!10}
\setbeamertemplate{footline}{%
  \leavevmode%
  \hbox{%
    \begin{beamercolorbox}[wd=.3\paperwidth,ht=2.25ex,dp=1ex,center]{author in head/foot}%
      \usebeamerfont{author in head/foot}\insertshortauthor\hspace*{1ex}(\insertshortinstitute)
    \end{beamercolorbox}%
    \begin{beamercolorbox}[wd=.2\paperwidth,ht=2.25ex,dp=1ex,center]{title in head/foot}%
      \usebeamerfont{title in head/foot}\insertshorttitle
    \end{beamercolorbox}%
    \begin{beamercolorbox}[wd=.4\paperwidth,ht=2.25ex,dp=1ex,center]{date in head/foot}%
      \insertshortdate{} @ \InsertConference
    \end{beamercolorbox}%
    \begin{beamercolorbox}[wd=.1\paperwidth,ht=2.25ex,dp=1ex,center]{page number in head/foot}%
      \insertframenumber{} / \inserttotalframenumber\hspace*{1ex}
    \end{beamercolorbox}}%
  \vskip0pt%
}

\beamertemplatenavigationsymbolsempty

\setbeamertemplate{bibliography item}{\insertbiblabel}
\setbeamersize{description width=1cm}
\setbeamertemplate{items}[circle]
\setbeamertemplate{section in toc}[circle]
\setbeamertemplate{subsection in toc}{%
  \leavevmode\leftskip=2em
  {%
    \usebeamerfont*{itemize item}%
    \usebeamercolor{subsection number projected}%
    \color{bg}%
    \raise1.25pt\hbox{\donotcoloroutermaths$\bullet$}}%
  \hskip1.5ex\inserttocsubsection\par}

% Definitions for the title page
\newcommand*{\GitHub}[1]{%
  \gdef\InsertGitHub{#1}%
}
\newcommand*{\Email}[1]{%
  \gdef\InsertEmail{\href{mailto:#1}{#1}}%
}
\newcommand*{\Conference}[1]{%
  \gdef\InsertConference{#1}%
}
\setbeamerfont{title}{size=\huge, series=\bfseries, family=\mcfamily\rmfamily}
\setbeamercolor{title}{bg=white}
\setbeamerfont{subtitle}{size=\small, series=\mdseries, family=\mcfamily\rmfamily}%\gtfamily\sffamily}
\setbeamerfont{email}{size=\scriptsize, family=\ttfamily}
\setbeamercolor{email}{bg=white}
\setbeamerfont{date}{shape=\itshape, family=\rmfamily}
\setbeamerfont{vc}{size=\scriptsize, family=\ttfamily}
\setbeamercolor{vc}{bg=white}

\renewcommand{\figurename}{Fig}

\input{vc.tex}

\setbeamertemplate{title page}
{%
  \vbox{}
  \vfill
  \begingroup
    \centering
    \hrulefill\par%
    \vskip1ex\par%
    \begin{beamercolorbox}[sep=0pt,center,shadow=false,rounded=true]{title}
      \vfill
      \usebeamerfont{title}\inserttitle\par%
      \ifx\insertsubtitle\@empty%
      \else%
        \vskip0.5ex%
        {\usebeamerfont{subtitle}\usebeamercolor[fg]{subtitle}\insertsubtitle\par}%
      \fi%
      \vfill  
    \end{beamercolorbox}%
    \hrulefill\par%
    \vskip2ex%
    \begin{beamercolorbox}[sep=0pt,center,shadow=false,rounded=true]{author}
      \usebeamerfont{author}\insertauthor
    \end{beamercolorbox}
    \begin{beamercolorbox}[sep=0pt,center,shadow=false,rounded=true]{email}
      \usebeamerfont{email}\InsertEmail
    \end{beamercolorbox}
    \vskip0.1ex
    \begin{beamercolorbox}[sep=5pt,center,shadow=false,rounded=true]{institute}
      \usebeamerfont{institute}\insertinstitute
    \end{beamercolorbox}
    \begin{beamercolorbox}[sep=5pt,center,shadow=false,rounded=true]{date}
      \usebeamerfont{date}\insertdate \normalfont @ \InsertConference
    \end{beamercolorbox}
    \begin{beamercolorbox}[sep=0pt,center,shadow=false,rounded=true]{vc}
      \usebeamerfont{vc}
      \url{https://github.com/\InsertGitHub} (\texttt{\GITAbrHash})
    \end{beamercolorbox}
    % {\centering
    %   \href{https://creativecommons.org/licenses/by-nc/4.0/}{%
    %     \includegraphics[width=0.1\textwidth]{img/by-nc.pdf}%
    %   }%
    % }
    {\usebeamercolor[fg]{titlegraphic}\inserttitlegraphic\par}
  \endgroup
  \vfill
}
\setbeamertemplate{blocks}[rounded][shadow=false]

% ============ ここを消すとNote消える ================
% \mode<handout>{%
%   \usepackage{pgfpages}
%   \setbeameroption{show notes on second screen=right}
%   \setbeamertemplate{note page}{%
%     \vspace{2ex}\insertnote%
%   }
% }
% ============ ここを消すとNote消える ================


\renewcommand{\kanjifamilydefault}{\gtdefault}

\setbeamertemplate{caption}[numbered]
\resetcounteronoverlays{lstlisting}
\definecolor{bluegray}{rgb}{0.4, 0.6, 0.8}
\DeclareCaptionFormat{listing}{{\color{bluegray}\lstlistingname}#2#3}
\captionsetup[lstlisting]{format=listing, font={footnotesize}}
\captionsetup[figure]{name={図}}
\captionsetup[table]{name={表}}
\setbeamerfont{footnote}{size=\scriptsize}

\setmonofont[Ligatures=TeX]{CMU Typewriter Text}

\setbeamertemplate{items}[circle]

\input{./lib/quotebox.tex}
\input{./lib/footnotemark.tex}
\input{./lib/ballon.tex}
\input{./lib/callout.tex}
\input{./lib/listings.tex}
\input{./lib/notes.tex}
\input{./lib/stack.tex}
\input{./lib/card.tex}

\newcommand\ce[1]{%
  \coloremojiucs{#1}
}

\newcommand*{\lstitem}[1]{
  \setbox0\hbox{\lstinline{#1}}
  \item[\usebox0]
}

\presetkeys{todonotes}{inline, noinlinepar}{}

\renewcommand{\arraystretch}{1.2}
\newcolumntype{Y}{>{\centering\arraybackslash}X}

\title[参照透過の実際]{%
  The Real World \\
  ``Referential Transparency''
}
\subtitle{参照透過の実際}
\author[吉村 優]{%
  吉村 優(\textsc{Yoshimura} Hikaru)
}
\Email{yyu@mental.poker}
\date[June 9, 2024]{%
  \oldstylenums{June 9, 2024}
}
\Conference{ScalaMatsuri 2024}
\institute[\InsertEmail]{%
  %株式会社リクルート(Recruit Co., Ltd) \\
  %\includegraphics[width=3cm]{./img/6_Brandlogo_2_Color.jpg}
}
\GitHub{y-yu/referential-transparency-slide}

\newcommand{\facesize}{1cm}
\newcommand\alicecallout[2]{
  \simplecallout[{\includegraphics[width=\facesize]{./img/alice_face.png}}]{#1}{cyan!10}{#2}
}
\newcommand\bobcallout[2]{
  \simplecallout[{\includegraphics[width=\facesize]{./img/bob_face.png}}]{#1}{orange!10}{#2}
}

\begin{document}

\frame{\maketitle}

\begin{frame}
  \frametitle{目次}

  \tableofcontents
\end{frame}

\section{参照透過とは?}

\begin{frame}[fragile]
  \frametitle{参照透過}

  \begin{itemize}
    \item<+-> \emph{参照透過(Referential transparency)}とは、式の構成要素がすべて同じなら、式の値は常に同じになる\cite{referential_transparency_def}

    \item<+-> たとえば下記のコードは参照透過になる
    \begin{figure}[h]
      \begin{columns}
        \begin{column}{0.4\textwidth}
\begin{lstlisting}[style=scala]
val two = 1 + 1
two
two
\end{lstlisting}
      \end{column}
      \begin{column}{0.4\textwidth}
\begin{lstlisting}[style=scala]
1 + 1
1 + 1
\end{lstlisting}
        \end{column}
      \end{columns}
      \label{fig:lst_referential_transparency}
      \caption{参照透過なコード例}
    \end{figure}

  \end{itemize}
\end{frame}

\begin{frame}[fragile]
  \frametitle{参照透過}

  \begin{itemize}
    \item<+-> 一方で下記のコードは参照透過\textbf{ではない}
    \begin{figure}[h]
      \begin{columns}
        \begin{column}{0.4\textwidth}
\begin{lstlisting}[style=scala]
val hello = {
  println("hello")
  "hello"
}
hello
hello
\end{lstlisting}
        \end{column}
        \begin{column}{0.4\textwidth}
\begin{lstlisting}[style=scala]
{
  println("hello")
  "hello"
}
{
  println("hello")
  "hello"
}
\end{lstlisting}
        \end{column}
      \end{columns}
    \end{figure}

    \item<+-> この左のコードは\lstinline|hello|の出力が1回だが、
    一方で右のコードは2回出力される
  \end{itemize}
\end{frame}

\begin{frame}[fragile]
  \frametitle{Scalaでの参照透過}

  \begin{itemize}
    \item<+-> 今みた\lstinline|println|のように、入出力は参照透過ではない
    \begin{itemize}
      \item 他にも時計へのアクセスとか
    \end{itemize}

    \item<+-> \lstinline|Future|もインスタンシエイト時にスレッドが走ってしまうので
    参照透過ではない
    \begin{columns}
      \begin{column}{0.4\textwidth}
\begin{lstlisting}[style=scala]
val f1 = Future(/* なにか */)
val f2 = Future(/* なにか */)
f1.flatMap(_ => f2)
\end{lstlisting}
    \end{column}
    \begin{column}{0.4\textwidth}
\begin{lstlisting}[style=scala]
Future(/* なにか */)
  .flatMap(_ => Future(/* なにか */))
\end{lstlisting}
      \end{column}
    \end{columns}
    \begin{itemize}
      \item この\ce{:point_up:}コードは異なる意味になる
    \end{itemize}
  \end{itemize}
\end{frame}

\begin{frame}[fragile]
  \frametitle{Scalaでの参照透過}

  \begin{itemize}
    \item<+-> 一方でMonixの\lstinline|Task|\cite{monix_task}は参照透過

    \item<+-> Extensible Effects(Eff)も参照透過をやっていくことが前提(?)
  \end{itemize}
\end{frame}

\section{このディスカッションの内容}

\newcommand{\thinking}{%
  {\large\ce{:thinking:}}
}
\begin{frame}[label=discussion]
  \frametitle{このディスカッションの内容}

  \pause
  \simplecallout[\thinking]{+}{cyan!10}{{\large 参照透過だと何がいいのか?}}

  \pause
  \simplecallout[\thinking]{-}{green!10}{{\large 参照透過どれくらいがんばるのか?}}

  \pause
  \simplecallout[\thinking]{+}{red!10}{{\large 参照透過は分かりやすいか分かりにくいか?}}

  \pause
  \simplecallout[\thinking]{-}{blue!10}{{\large 参照透過は他の言語でどうか?}}  
\end{frame}

\section{登壇者}

\begin{frame}
  \centering
  \mcfamily\rmfamily\bfseries{\fontsize{50pt}{50pt}\selectfont 登壇者}%
\end{frame}

\begin{frame}
  \begin{columns}
    \begin{column}{0.45\textwidth}
      \begin{center}
        \begin{figure}
          \includegraphics[width=0.2\textwidth]{img/kmizu.png}
        \end{figure}
      \end{center}
 
      \begin{table}[h]
        \begin{tabular}{ll}
          Twitter & \href{https://twitter.com/kmizu}{@kmizu} \\
          GitHub &  \href{https://github.com/kmizu}{kmizu} \\
        \end{tabular}
      \end{table}
    \end{column}
    \begin{column}{0.45\textwidth}
      \begin{center}
        \begin{figure}
          \includegraphics[width=0.2\textwidth]{img/kory.png}
        \end{figure}
      \end{center}
 
      \begin{table}[h]
        \begin{tabular}{ll}
          Twitter & \href{https://twitter.com/kory\_\_3}{@kory\_\_3} \\
          GitHub &  \href{https://github.com/kory33}{kory33} \\
        \end{tabular}
      \end{table}
    \end{column}
  \end{columns}

  \begin{columns}
    \begin{column}{0.45\textwidth}
      \begin{center}
        \begin{figure}
          \includegraphics[width=0.2\textwidth]{img/bird2x.png}
        \end{figure}
      \end{center}
 
      \begin{table}[h]
        \begin{tabular}{ll}
          Twitter & \href{https://twitter.com/\_yyu\_}{@\_yyu\_} \\
          GitHub &  \href{https://github.com/y-yu}{y-yu} \\
        \end{tabular}
      \end{table}
    \end{column}
    \begin{column}{0.45\textwidth}
      あと他にもだれか(?)
    \end{column}
  \end{columns}
\end{frame}

\againframe{discussion}

\section*{参考文献}
\begin{frame}%[allowframebreaks]
  \frametitle{参考文献}
  % \nocite{*}
  \bibliographystyle{junsrt_url}
  \bibliography{ref}
\end{frame}

\begin{frame}
  \centering
  {\Huge Thank you for the attention!}
\end{frame}

\end{document}
